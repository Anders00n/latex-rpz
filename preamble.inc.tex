\sloppy
% \fussy

% Настройки стиля ГОСТ 7-32
% Для начала определяем, хотим мы или нет, чтобы рисунки и таблицы нумеровались в пределах раздела, или нам нужна сквозная нумерация.
\EqInChapter % формулы будут нумероваться в пределах раздела
\TableInChapter % таблицы будут нумероваться в пределах раздела
\PicInChapter % рисунки будут нумероваться в пределах раздела

% Добавляем гипертекстовое оглавление в PDF
\usepackage[
    bookmarks=true, colorlinks=true, unicode=true,
    urlcolor=black,linkcolor=black, anchorcolor=black,
    citecolor=black, menucolor=black, filecolor=black,
]{hyperref}

% Изменение начертания шрифта --- после чего выглядит таймсоподобно.
% apt-get install scalable-cyrfonts-tex

\IfFileExists{cyrtimes.sty}
{
    \usepackage{cyrtimespatched}
}
{
    % А если Times нету, то будет CM...
}

\usepackage{graphicx}   % Пакет для включения рисунков

% С такими оно полями оно работает по-умолчанию:
% \RequirePackage[left=20mm,right=10mm,top=20mm,bottom=20mm,headsep=0pt]{geometry}
% Если вас тошнит от поля в 10мм --- увеличивайте до 20-ти, ну и про переплёт не забывайте:
\geometry{right=20mm}
\geometry{left=30mm}
\geometry{bottom=20mm}
\geometry{ignorefoot}% считать от нижней границы текста

% \graphicspath{{figures/}}

\usepackage{microtype}

% Пакет Tikz
\usepackage{tikz}
\usetikzlibrary{arrows,positioning,shadows}

% Произвольная нумерация списков.
\usepackage{enumerate}

% ячейки в несколько строчек
\usepackage{multirow}

% itemize внутри tabular
\usepackage{paralist,array}

%разраяженные буквы
\usepackage{soulutf8}

%вставка pdf-страниц
\usepackage{pdfpages}

%супер-пупер таблицы
\usepackage{tabularx}

%подпись к уравнениям
\usepackage{eqexpl}
\eqexplSetIntro{где}
\eqexplSetSpace{3mm}

\usepackage[normalem]{ulem}

\usepackage{wrapfig}

\usepackage{setspace}

\usepackage{booktabs}

\usepackage{rotating}

\usepackage{makecell}

\usepackage{adjustbox}

\usepackage{xltabular}

\usepackage{lscape}

% \usepackage{sidewaystable}

\newcolumntype{Y}{>{\centering\arraybackslash}X}
\newcolumntype{C}[1]{>{\Centering\arraybackslash}p{#1}}

% \usepackage{newfloat}
% \DeclareFloatingEnvironment[
% placement={!ht},
% name=Equation
% ]{eqndescNoIndent}
% \edef\fixEqndesc{\noexpand\setlength{\noexpand\parindent}{\the\parindent}\noexpand\setlength{\noexpand\parskip}{\the\parskip}}
% \newenvironment{eqndesc}[1][!ht]{%
%     \begin{eqndescNoIndent}[#1]%
% \fixEqndesc%
% }
% {\end{eqndescNoIndent}}

% \setlength{\parskip}{1ex} % разрыв между абзацами
% \usepackage{blindtext}
